Associate editor comments:

Both reviewers found this manuscript important and interesting. Both offer some critiques on modeling choices that must be addressed in a revision. Please also pay close attention to reviewers comments about areas where statements may not fully reflect current literature and where the manuscript overstates its novelty.



Reviewer comments:


Reviewer 1: First, I would like to commend the authors for addressing a vital topic in their research. Research into how the sentiment-related effects of the weather differ by demographic status is very important as it relates to potential adaptation and resilience strategies. The authors have put together a very nice manuscript, and it was a pleasure to read.

That said, I think this is a solid manuscript that isn't quite ready for publication and needs some minor and some major tweaks. I have a number of comments, questions, and concerns that should be addressed prior to publication. I think once these concerns are addressed this manuscript will be worthy of publication.

Comments:

Comments follow in order of appearance in the manuscript.

1. The abstract is a bit incorrect. The authors cite a number of studies that, themselves, have sections where those studies examine heterogeneous impacts as a function of demographic vulnerability - even individually reported income. There's no need to over-sell their novelty on this point. The authors should mention/detail the heterogeneous impacts already well-documented in the climate/mental health literature, the prior effect size differences, alongside whether those differences were significant or not. Simply stating 'heterogeneous tests haven't been done or haven't seen anything' is incorrect.
% #Wording: Over-selling "first evidence". Change the last sentence of abstract.

2. "Expressed mood" or "mood" represents a sizable conceptual leap from 'sentiment' to affective states. If the authors want to argue that their construct measures affective states, they will need to justify this measurement leap a bit more strongly. To my knowledge the link between 'sentiment' and 'mood' is still only weakly validated, at best.
%#Major/#Minor: Leap or link between mood and sentiment is weak. Need additional literature review.

3. That "previous analysis [sic] using city-level data have been unable to find heterogeneities" isn't correct. The authors need to tone down the strength of this claim and better incorporate prior heterogeneous findings in the social impacts literature into their introduction and discussion. E.g. from Obradovich et al., 2018:

"To examine if poorer respondents are more sensitive to temperature, we stratify our sample along income quartiles and estimate regressions for the lowest and highest quartiles (SI Appendix, Table S8). Fig. 2A shows that the negative effect of temperatures greater than 30 °C on the probability of mental health issues is larger for low-income respondents (coefficient: 2.309, P = 0.002, n = 438,518). This is 1.6 times the effect observed among the highest-income adults in the sample (coefficient: 1.458, P = 0.005, n = 509,608).

Fig. 2B shows that the negative effect of temperatures greater than 30 °C on the probability of mental health difficulties is largest for women (coefficient: 1.41, P = 0.005, n = 1,211,220). This is 1.6 times the effect observed among men in our sample (coefficient: 0.879, P = 0.031, n = 750,523). Combining these insights, the effect observed in the subsample of low-income women (coefficient: 2.742, P = 0.002, n = 300,570) is approximately two times the magnitude of the effect observed in the high-income men in the sample (coefficient: 1.373, p = 0.03, n = 235,098)."
%#Wording: Change the tone for previous analysis 
4. "We present the first evidence…" can also be toned down.
%#Wording: Change the saying for overall novelty

5. There are a number of typos throughout the manuscript. Another proof read could be useful.
%#Minor: Check the typos

6. We've found the same sleep effects within actigraph-based data across the diurnal cycle, as well: https://arxiv.org/abs/2011.07161 (we also find quite substantial country-level heterogeneity across income in the effects of temperature on sleep).
%#Minor: Need additional literature review
7. "research conducted at national scales indicates surprisingly little heterogeneity in impacts" simply isn't consistent with my understanding of the literature. Some literature doesn't find significant differences, others find sizable heterogeneity.
%#Minor / #Wording: Need to review and summarise literature again
8. You cite, in part, Mullins and White's use of individual BRFSS data to justify the statement that a major challenge is that the data are aggregated. The survey measures used by Mullins and White and Obradovich et al. contain more precisely measured demographic covariates than the ecological inference approach you employ here…
% #Dispute: The aggregation level is the difference.

10. Zheng et al. 2020 examine by-city heterogeneous effects of income as well as gender on the temperature-sentiment relationship. Given the extensive literature on gender and heterogeneous mental health, the authors might do well to discuss this dimension of vulnerability, as well (though I understand their Twitter data does not easily enable investigation in this setting; Weibo data enabled more precise measurement). https://www.sciencedirect.com/science/article/pii/S2590332220302529#fig3

Further, Mullins and White found sizable differences by gender, see their Table A12.
% #Minor: Need additional literature review
11. The whole introduction is framed around a set of findings that is not nicely consistent with the literature as I understand it. There's no need to motivate this paper as the first paper to investigate heterogeneous effects in climate/mental health. The effects documented are interesting and important in their own right.
% #Wording: Change the tone of "first paper"

12. That many studies use sentiment as a measure of mood does not mean the measurement leap is well-validated. I wish it were better validated, but am not convinced that it yet is and have been convinced otherwise by some expert psychometricians.
% #Major: Change the usage of "mood" to "sentiment" See 2.
13. Jumbling up sentiment into the various Twitter-mental health papers is coarse. There are quite a large number of differences between those studies that try to derive features predictive of mental health outcomes and the use of sentiment as a direct proxy for those outcomes. This should be made more clear.
% #Major: Reorg the part of literature of Twitter-mental health
14. How did the authors define 'weather-related words'? It would be good to state this in the manuscript.
% #Minor: Add weather-related words into Appendix?

15. "Expressed mood is a widely used metric to assess mental health and well-being from posts on social media platforms." You are speaking of sentiment here, not 'expressed mood'. If you want to argue that one is a valid measure for the other, you will need to make a more compelling argument on that front. That others make the leap, by itself, isn't particularly compelling to me.
% #Major: Change the usage of "mood" to "sentiment". See 2 and 12.
16. One of the biggest conceptual/measurement leaps that the authors ask the reader to make in this manuscript is that neighborhood characteristics map to the individuals who post geolocated tweets. This is a very large conceptual leap. We know that a) Twitter users are not representative of the general public and b) those who geolocate their tweets are not representative of Twitter users. To map census characteristics onto these users - in the primary rationale for this entire paper - requires convincing the reader that such census characteristics do an adequate job serving as a proxy for individual geo-tweet users' characteristics. The authors should conduct more extensive testing and validation to justify this measurement leap. Noting it merely as a limitation in the discussion isn't sufficient.
% #Major: Do test to show the representative of geo-located Twitter is good enough to be mapped by census characteristics. May need literature review on twitter related papers

17. The cold side of the functional form presented by the authors in Fig. 1 does not map well to any prior result on this topic in the literature that I am aware of, which is concerning. Every other published paper that I'm aware of on this topic, using much the same data, finds a decline in sentiment on the cold side of the distribution (Baylis et al 2018, Baylis 2020, Wang et al 2020, etc.). (The global functional form also had a cold-side effect in this study: https://www.thelancet.com/journals/lancet/article/PIIS0140-6736(21)01787-6/fulltext)
% #Minor: Comparing the cold side difference from other paper.

18. Looking into the methods that might produce this different functional form, I see that the authors do not include space-by-time fixed effects that are standard in the climate econometrics literature. Why is this? The authors' primary functional form - and model - should match that of the literature in which they are attempting to investigate heteregeneous effects. Space-by-time fixed effects are an important way to rule out confounding variation that varies across spatial units (counties?) over time. I strongly suggest the authors modify their empirical approach accordingly.

Understanding the heterogeneous effects that might exist on the well-documented cold side of the distribution is just as important as understanding how these effects operate on the hot side of the distribution, if the goal is to understand the net-impact of warming across heterogeneous demographics.
% #Major: Rerun regression model with space-time fixed effect
19. Why do the authors not extend model (3) to the cold side of the temperature distribution? It is just as important to understand time-of-day effects on that side of the distribution, too (even within the relatively flat effects they observe on average, likely due to omission of fixed effects important in this type of modeling, there could easily be within-day effects that average out, right?). I'm fine if they model that side as a linear function of temperature, too, but not that they exclude it entirely.
% #Major: Rerun regression model for cold side
20. Income is a continuous variable and the goal of this study is to document heterogeneous effects. The authors choose two income categories to compare to one another, excluding 90\% of the income distribution. What happens across the rest of the income distribution? Are the heterogeneous effects mostly linear as they change across the distribution? Are their nonlinearities? This is critical to investigate, and the authors have the data to do so.
% #Major: Income group breakdown
21. Similarly, the authors' racial categories are continuous in measurement, but they for some reason choose to dichotomize them. This isn't really appropriate if the primary goal of this manuscript is to understand the nature of heterogeneous effects. The authors should present effects across the entire distribution of vulnerability, or at least show us how their results persist across more flexible and continuous definitions of their vulnerability measures.
% Zheng: I didn't get why racial categories are continuous
22. The by-time-of day analysis draws strange inference. We know a) that the effect of temperature on mood is curved and nonlinear and b) that it appears to be accelerating in hotter temperatures. That warming in early morning produces positive effects and in the afternoon produces negative effects could simply be an artifact of the absolute-location on the temperature distribution held by those time periods, on average. I would, instead, like to know: For any given time of day, how does the effect of temperature - across the distribution of all temperatures that occurred in that hour over the sample - compare to any other time of day. The authors could choose times of day with decent support in the data to compare, running a regression for each hour of the day and plotting them on the same axes.
% #Major: Change the display of Time of Day breakdown
23. "Previous work conducted with county-scale data found no heterogeneity in mental health vulnerability by income." is of questionable accuracy.
% #Minor: Need additional literature review.
Overall I think the import of this study is high (no need to oversell it!) and the execution is solid. With the addressing of the above concerns I think it will be a useful addition to the literature.






Reviewer 2: The manuscript tackles an important climate change health impact of mental well-being and according to vulnerability of race and income using an innovative way of extrapolating Twitter data regarding expressions of mood with temperatures and socioeconomic data according to census blocks.
The manuscript is well written and is addressing neglected areas of climate change health impacts. There are some important findings regarding how increased temperatures are associated with lower mood as expressed by a large number of people through 242 million tweets across the United States, and especially among Black Americans. The authors carefully describe the methods and modelling and include covariates of time of day, day of the week, and the day in the year, rain, sunshine, as well as income and ethnicity.
There are the following issues that weaken the manuscript and its potential interpretations:
1. Without inclusion of air conditioning status/prevalence within each of the census blocks, or if not available, at least within the areas that such data is available as secondary analyses, the findings of the association between mood and heat can be easily explained by the lack of air conditioning rather than the other factors of race, income, time of the data, day of the week, etc.
% #Major  #Improvement #If-Applicable: introduce air conditioning with the modeling
2. There is a discussion to make the case that mood as expressed in Twitter equals mental health and mental illness (like lines 345-348 in the discussion), but this is a bit of stretch. For example, Reece et al uses an algorithm model to predict PTSD but cautions that this is just a proof of concept and not to be used before further research and refinement. The VADER sentiment analyses authors do not claim their scoring has anything to do with mental illnesses.
% #Minor #Wording: Relation between mood and mental health. See Reviewer1:2,12,15
3. The use of tweets from 2014 to be correlated to socioeconomic and racial distribution of the census blocks from the American Community Service data during 2012-2016 but then tweets after 2016 correlated with American Community Service data during 2014-2018 was not consistent and there is a possibility of changes in these populations or how they tweet their mood given twitter is a relatively recent and growing phenomena that became more common after 2014.
% #Major  #Improvement: Add discussion between using two ACS datasets
4. There is an assumption that impacts of heat on mood is immediate, while in general the longer the heat wave the more likely it will impact mental health, so a lag of several days or limiting the analyses to heat waves of several days would have been more appropriate.
% #Major #Improvement: Consider the time lag of heat wave effect
5. Similarly, heat during the night is more important than heat and high temperature during the day. This was evident from the fact that those who tweet at 6am are more likely to have a bad mood. Carrying out the analyses only for high temperatures during the night compared to any time of the day would shed more light on this matter.
% #Major #Improvement: time of day breakdown; See Reviewer1:22
6. Although in the supplement data there is a graph showing the different metropolitan cities across the nation all had negative mood tweets during higher temperature, it is a well-known fact that there is wide variety and range of tolerance to heat waves and higher temperature in places like Arizona, Texas, or New Mexico compared to the Northeast or Midwest. In our work we even found within the same cities those living in the coastal area exposed to the sea breeze are less tolerant to heat waves than those inlands who are more used to higher average temperatures. This matter can be addressed by carrying out the analyses with different cutoff/range of dry bulb for each region in the US.
% #Major #Improvement: cities trend groupby dry bulb
7. Similar to point 6, it is not clear why those living in high socioeconomic status area have better mood with higher temperature than those with low income status but then at higher temperature their mood declines. If those with high socioeconomic status have air conditioning then they would not be impacted by higher temperature. If both groups do not have air conditioning then one would expect that those with high income are less tolerant than the low socioeconomic status group.
% Zheng: I didn't see why wealthy people would be totally tolerant to high temperature.
8. I presume the tweets do not differentiate individuals who carry out multiple tweets, but this was not addressed. If someone is a very active tweeter and engaged with many on twitter, they will likely bias the results. If this data is available (tweets per individual) maybe do a weighting for such tweets from them same source, and if the data is not available then this should be addressed in the discussion.
% #Major #If-Applicable: Unique users problem
9. Given the nature of this data from twitter and the many assumptions in attributing how personal expressions of a mood can be related to heat at an ecological census block level, I would not agree with the concluding remarks and the certainty that heat waves will impact the mental health of low-income populations. It is a bit of a leap. This is especially true given past studies have not shown this differentiation, and the above mentioned issues with the data lacking air conditioning rates, and the counterintuitive higher tolerance to heat among the higher income populations, all weaken interpretations of the results from this study.

% #Major: Limitation of using Twitter weaken the conclusion


Requests from EHP

Note to authors: EHP conducts content editing during peer review as a service to authors and an internal quality assurance measure. Please verify that the revised version of your manuscript addresses the following comments before submitting your revision (also attached). In addition, please see attached reporting guidelines for EHP research articles. Doing so may avoid the need for additional revisions that could delay final decisions and publication.

- Line 40, Add qualifier, “first, to our knowledge” or similar.
- Line 153: Is there any knowledge of the % of this sample of tweets from the total number of geolocated tweets in this area/time frame? If so, please specify.
- Was there any linkage to individual users, or were the tweets sample and analysis limited to an analysis unit of a tweet (for example, was this a random sample of tweets or was there sampling/clustering by user)? Please clarify.
- Does this include only public tweets/accounts, or all tweets/accounts? Please specify.
- Line 175: Is there an upper or lower bound to the VADER score? If so, please include the bounds here.
- Line 176-178. Provide a brief description of the Hedonmeter and AFINN methods. Please also include some discussion of why the VADER method was used for the primary analysis over the other available methods (is one shown to be more valid than the other, is one more widely used, etc?).
- Please confirm if the analysis accounted for region, as response to temperature can vary widely due to acclimatization. For example 90 degrees in Seattle can elicit a different response than 90 degrees in Los Angeles. This concern was also brought up by reviewer 2.
- Clarify in method how the census block majority was calculated.
- Justify why you collapse race groups (line 215-216, line 227-230). Add to discussion, the limitations of this classification and it’s use as a proxy for unmeasured consequences of racism. Please see EHP guidelines on the use and reporting of race/ethnicity in research articles: https://ehp.niehs.nih.gov/authors/research-articles.
- Add to discussion, how air conditioner availability by region may impact results. Consider a secondary analysis including % population with air conditioning (there is public data on this available). This issue was also brought up by reviewer 2.
- Add to discussion, are there other secular trends over time that might impact results?
- Line 251: Typo, “were” should be “where”. Please also correct other typos throughout.
- Please describe all secondary analyses in the methods section, and refer to results of these analyses in the results section. Line 299 and 324.
- Figure 1: Unclear what “tweet count” on the y-axis refers to. Are there 2 sets of data presented on the same graph? This is not clear and there is no unit note about the tweet count. Please clarify on the figure and in the figure caption.
- Please refer to specific material in the supplement rather than “see supplement”. Such as (Figure S1.), or (Supplement, Title of section in the supplement)
- Figure 1, 2: Include in figure caption description of y and x-axes, make sure it is clear two sets of information are presented on the same graph.
- Figure 2: Change so both panel a and b use the same y axis
- Include definition of “poorest” and “wealthiest” neighborhoods in methods section.
- In fig 3 you use poor/rich, in the text you use wealthiest/poorest – please use consistent wording.
- Add to methods percentile cut points for income for analysis.
- Where there any census blocks with missing ACS data? If so, how was that handled in the analysis?
- Please include in the results, the coverage statistics of where tweets were located (% census blocks or census tracts in the US). Consider adding a jittered map of locations. Were certain areas or regions in the US more represented than others?
- Please present the number/proportion of total tweets by the sub-populations you examine.
- Add to methods mention of the comparison to natural disasters. Methods for all results presented in results should be described in the methods section. How was Hurricane Sandy chosen for comparison? How were “impacted counties” defined? What was the size for the tweets included in the analysis for Hurricane Sandy.
- Add to methods a description of the Saturday to Monday comparison.
- Add to methods, comparison with CDC WONDER data
- Line 367-369 – what is this based on? Include a citation.
- Line 418 – include as standard citation, not as a hyperlink
- Please add to discussion what the change in VADER score might represent (how do you interpret a .5 change, a .05 change?).
- Please include tables with corresponding results for figures in the supplement for transparency.
- Figure S7 – Include \$ associated with presented tertiles.
- In supplemental material, the temp effects by CBSA needs to be mentioned in the main text. Same with interaction of race and income.
- Please ensure all material in the supplement is specifically called-out in the main text. For example,
call out weather terms supplement material in methods.
- EHP submission format: Please provide main text figure captions on a new page after any tables (which should be on a new page after the references).
- Revise all current findings, interpretations, and conclusions to be past tense rather than present
- Note that EHP is required by law to conform to 508 standards for accessibility, which provides guidelines to ensure that the content is accessible in some form to all readers, including those with color vision disabilities. In general, do not identify elements of a figure by color alone, as individuals with color vision disabilities will not be able to discern that color. Instead, use contrast, texture, or symbols. Color may be used to enhance figures, but to ensure that figures are accessible to readers with impaired color vision, color should not be used as the only means of conveying information unless absolutely necessary. Use different symbols, shading/textures, and line patterns instead of (or in addition to) color to distinguish among different data points, or use contrasting colors that can be easily distinguished from each other when the figure is printed in black and white. Alternatively, select a “colorblind palette” if available in the software package being used to generate the figure. If colors are used to identify data, include a color key in the graphical portion of the figure: do not refer to colors by name (i.e., red, blue) in the figure title or caption. For example: Figure 2.
- Revise figure captions: Tables and figures do not have to be 100\% stand-alone, but they should include enough information to be easily interpreted by a casual reader who may be “skimming” the manuscript to decide whether it is worth a more detailed assessment. This applies to tables and figures provided in the separate Supplemental Material file, as well as tables and figures in the main text. If you are in doubt about the amount of information to include, please err on the side of excess information.
o Provide brief information about the study population (including location, overall size, and time period), usually in the table or figure title.
o Tables or figures reporting model-based estimates should include brief information regarding the methods used to derive the estimates, including the type of model, covariates, missing data methods (e.g., imputation), and methods to account for values o Report numbers of observations used to derive each individual estimate reported in the table or figure, after accounting for missing data and subgrouping. As appropriate, indicate numbers of observations by subgroup. (These can be provided in the supplemental material)