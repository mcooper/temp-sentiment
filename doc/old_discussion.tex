% Random discussion paragraphs that were cut

% Comment on interpretation of results
We examined our results relative to an optimal baseline temperature of 5\textdegree C WBGT (12\textdegree C/54\textdegree F), as this was the temperature at which the highest sentiment scores were observed in the aggregated analysis (See Fig. \ref{fig:wbgt}).  However, after examining heterogeneities by neighborhood race and income, we find different optima for different racial groups and income levels.  For example, the optimum temperature for poor neighborhoods is 0\textdegree C WBGT (6\textdegree C/43\textdegree F) while the optimum temperature for rich neighborhoods is 20\textdegree C WBGT (29\textdegree C/84\textdegree F).  These different varying optima show that even mild heat can affect human well-being, and the only reason that sentiment peaks as high as 20\textdegree C WBGT (29\textdegree C/84\textdegree F) in rich neighborhoods is because the rich are more likely to have air conditioning in their homes, transportation, and work spaces, as well as much greater choice in when they are outside and in the activities they perform outside.
